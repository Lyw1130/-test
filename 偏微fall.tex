\documentclass{report}
\parindent=0pt
\usepackage[margin=2cm]{geometry}
\usepackage{amsmath}
\begin{document}


\paragraph{Problem A.} Let \textit{I}  =[0,2$\pi$] and \textit{N} > 0 be a positive interger. 
We partition \textit{I} by \\
\begin{equation}
x^{}_{i} = \textit{ih} , \textit{h} = \frac{2\pi}{N+1},  \textit{i} = 0,1,\dots,N + 1, \label{1}
\end{equation}


where \textit{h} is spatial step size.\\


\qquad Consider a 2-$\pi$ periodic function \textit{f} . Let $f^{}_{i}$ = \textit{f}($x^{}_{i}$) and $\textit{f}^{\ \prime}_{i}$ $\approx$ \textit{df/dx}${\vert} ^{}_{x^{}_{i}}$. A student designs a difference approximation to derivative of \textit{f} as follows,
\begin{equation}
\frac{1}{4}\textit{f}^{\ \prime}_{i+1}+\textit{f}^{\ \prime}_{i}+\frac{1}{4}\textit{f}^{\ \prime}_{i-1} = \frac{3}{2}\frac{(\textit{f}^{}_{i+1}-\textit{f}^{}_{i-1})}{2\textit{h}}.\label{2}
\end{equation}
\paragraph{Question}: Analyze the truncation error of the approximation decribed in Eq(2).\\
\noindent\rule[0.25\baselineskip]{\textwidth}{1pt}

\paragraph{Solution}: \\
\\
We know that truncation error is the absolute value of difference between differentives and finite differences at $x^{}_{i}$ :

\begin{equation}
\textit{T.E.} =\left \vert \frac{df(x^{}_{i})}{dx} -f^{ \prime}_{i}\right\vert\label{3}
\end{equation}
By Taylor series, we can write $f^{}_{i+1}$ and $f^{}_{i-1}$ at $x^{}_{i}$ as follows :\\

\begin{equation}
f^{}_{i+1}=f^{}_{i} + hf^{\prime}_{i} + \frac{h^{2}}{2!}f^{\prime\prime}_{i} + \frac{h^{3}}{3!}f^{\prime\prime\prime}_{i}+ \dots \label{4}
\end{equation}
\begin{equation}
f^{}_{i-1} =f^{}_{i} - hf^{\prime}_{i} + \frac{h^{2}}{2!}f^{\prime\prime}_{i} - \frac{h^{3}}{3!}f^{\prime\prime\prime}_{i}+ \dots \label{5}
\end{equation}

Since that, we can write $f^{\prime}_{i+1}$ and $f^{\prime}_{i-1}$ at $x^{}_{i}$ by differentiation of (4) and (5) as follows :\\

\begin{equation}
f^{\prime}_{i+1}=f^{\prime}_{i} + hf^{\prime\prime}_{i} + \frac{h^{2}}{2!}f^{\prime\prime\prime}_{i} + \frac{h^{3}}{3!}f^{4}_{i}+ \dots 
\end{equation}

\begin{equation} 
f^{\prime}_{i-1} =f^{\prime}_{i} - hf^{\prime\prime}_{i} + \frac{h^{2}}{2!}f^{\prime\prime\prime}_{i} - \frac{h^{3}}{3!}f^{4}_{i}+ \dots 
\end{equation} 

By Eq(4) - Eq (7) , we can write
\begin{displaymath} 
f^{\prime}_{i+1} + f^{\prime}_{i-1}= 2(f^{\prime}_{i}+\frac{h^2}{2!}f^{\prime\prime\prime}_{i}+\frac{h^4}{4!}f^{5}_{i}+ \dots)
\end{displaymath} 
and
\begin{displaymath} 
f^{}_{i+1} - f^{}_{i-1} = 2(hf^{\prime}_{i}+ \frac{h^3}{3!}f^{\prime\prime\prime}_{i}+ \frac{h^5}{5!}f^{5}_{i} +\dots)
\end{displaymath} 
So we can write Eq(2) as,
\begin{align*}
f^{\prime}_{i} & = \frac{3}{2}\frac{(\textit{f}^{}_{i+1}-\textit{f}^{}_{i-1})}{2\textit{h}} - \frac{1}{4}(\textit{f}^{\ \prime}_{i+1}+\textit{f}^{\  \prime}_{i-1}) \\
& =\frac{3}{2} (f^{\prime}_{i}+ \frac{h^2}{3!}f^{\prime\prime\prime}_{i}+ \frac{h^4}{5!}f^{5}_{i} +O(h^6))-\frac{1}{2}(f^{\prime}_{i}+\frac{h^2}{2!}f^{\prime\prime\prime}_{i}+\frac{h^4}{4!}f^{5}_{i}+O(h^7))\\
&=f'_{i} +\frac{3}{2}*\frac{h^4}{5!}f^{5}_{i} - \frac{1}{2}*\frac{h^4}{4!}f^{5}_{i}+O(h^6)\\
&=f'_{i}- \frac{h^4}{5!}f^{5}_{i}+O(h^6)
\end{align*}

\newpage

Using the result, now we have 
\begin{align*}
\textit{T.E.}  & =\left \vert \frac{df(x^{}_{i})}{dx} -f^{ \prime}_{i}\right\vert\\
&=\left\vert \frac{df(x^{}_{i})}{dx} -(f'_{i}- \frac{h^4}{5!}f^{5}_{i}+O(h^6)) \right\vert\\
&=\left\vert \frac{h^4}{5!}f^{5}_{i} - O(h^6) \right\vert\\
\end{align*}
The answer of Pormble A is
\[
 \frac{h^4}{5!}f^{5}_{i} - O(h^6)
\]
\newpage
\paragraph{Problem B.} Consider the wave problem
\begin{equation}
\frac{\partial{u(x,t)}}{\partial{t}}	+ \frac{\partial{u(x,t)}}{\partial{x}} = 0 , x \in [0,2\pi] , t \ge 0 
\end{equation}
\begin{equation}
u(x,0) = f(x)
\end{equation}
where x and t are the space and time coordinates, respectively, and u and f are assumed two-$\pi$ periodic
functions.\\

\qquad To solve the wave problem numerically, we collocate the grid points as described in Eq. (1).Let n benon-negative integers. Denote the discrete time coordinates by tn = nk, where k is the time step.

\paragraph{Question.} : \\
\paragraph{1.}Construct a scheme based on discretizing $\partial{u}/\partial{t}$ by the central difference and approximating  $\partial{u}/\partial{x}$
by the difference method described in Eq. (2).\\

\paragraph{2.} What is the constraint on k and h such that the scheme can be performed stably?\\
\noindent\rule[0.25\baselineskip]{\textwidth}{1pt}
\paragraph{Solution 1.} :\\

We know
\begin{displaymath}
\frac{1}{4}\textit{f}^{\ \prime}_{i+1}+\textit{f}^{\ \prime}_{i}+\frac{1}{4}\textit{f}^{\ \prime}_{i-1} = \frac{3}{2}\frac{(\textit{f}^{}_{i+1}-\textit{f}^{}_{i-1})}{2\textit{h}}.
\end{displaymath}
\begin{displaymath}
u(x,0) = f(x)
\end{displaymath}
Then we let $u^{n}_{i}$ =   $u(x^{}_{i},t^{}_{n})$  \ ,\  $u^{}_{i}$ =$ f(x^{}_{i})$ , and put it into Eq(2) as follows,
\begin{displaymath}
 \frac{1}{4}\textit{u}^{\ \prime}_{i+1}+\textit{u}^{\ \prime}_{i}+\frac{1}{4}\textit{u}^{\ \prime}_{i-1} = \frac{3}{2}\frac{(\textit{u}^{}_{i+1}-\textit{u}^{}_{i-1})}{2\textit{h}}.
\end{displaymath}
\begin{equation}
 \textit{u}^{\ \prime}_{i+1}+4\textit{u}^{\ \prime}_{i}+\textit{u}^{\ \prime}_{i-1} = \frac{3}{h}(\textit{u}^{}_{i+1}-\textit{u}^{}_{i-1}).
\end{equation}
Now we have the recursion function of $u^{}_{i}$ , and by 2$\pi$ periodic we know $u^{}_{-1}$ = $u^{}_{N}$ \ , \ $u^{}_{N+1}$ = $u^{}_{0}$ , we can perform  Eq(10) as 
\begin{equation}
\left[ 
\begin{matrix}
4 & 1 & ~ & 1\\
1 & \ddots & \ddots & ~\\
~ & \ddots & \ddots & 1\\
1 & ~ & 1 & 4
\end{matrix}
\right] 
\left[ 
\begin{matrix}
u^{\prime}_{0}\\
u^{\prime}_{1}\\
u^{\prime}_{2}\\
\vdots\\
u^{\prime}_{N}\\
\end{matrix}
\right]
=
\frac{3}{h}
\left[
\begin{matrix}
0& 1 & ~ & -1\\
-1 & \ddots & \ddots & ~\\
~ & \ddots & \ddots & 1\\
1 & ~ & -1 & 0
\end{matrix}
\right]
\left[
\begin{matrix}
u^{}_{0}\\
u^{}_{1}\\
u^{}_{2}\\
\vdots\\
u^{}_{N}\\
\end{matrix}
\right]
\end{equation}
Let 
\begin{align*}
P = 
\left[ 
\begin{matrix}
4 & 1 & ~ & 1\\
1 & \ddots & \ddots & ~\\
~ & \ddots & \ddots & 1\\
1 & ~ & 1 & 4
\end{matrix}
\right] 
 \ , \ 
Q =
\left[
\begin{matrix}
0& 1 & ~ & -1\\
-1 & \ddots & \ddots & ~\\
~ & \ddots & \ddots & 1\\
1 & ~ & -1 & 0
\end{matrix}
\right]
 \ , \
\textbf{u}' = 
\left[ 
\begin{matrix}
u^{\prime}_{0}\\
u^{\prime}_{1}\\
u^{\prime}_{2}\\
\vdots\\
u^{\prime}_{N}\\
\end{matrix}
\right]
  \ , \ 
\textbf{u} = \left[ 
\begin{matrix}
u^{}_{0}\\
u^{}_{1}\\
u^{}_{2}\\
\vdots\\
u^{}_{N}\\
\end{matrix}
\right]
\end{align*}

\newpage
So Eq(11) can translate as
\begin{align*}
P \textbf{u}'=\frac{3}{h}Q \textbf{u}
\end{align*}
Since $P$ is a symmetric matrix and $P^{}_{ii}$ > 0 , for all $\textit{i}$  = 1 , 2 , \dots ,N , $P$ is a positive defined matrix,\\
so that $P^{-1}$ exists. 
And we recall $\partial{u}/\partial{x} \approx u^{\prime}$ , so we have
\begin{align*}
\frac{\partial{\textbf{u}}}{\partial{x}} \approx \textbf{u}^{\prime} = \frac{3}{h}P^{-1} Q \textbf{u}
\end{align*}
By the central difference we have $\partial{u}/\partial{t}$ as follows,
\begin{align*}
\frac{\partial{\textbf{u}}}{\partial{t}}\approx \frac{\textbf{u}^{n+1}_{i}- \textbf{u}^{n-1}_{i}}{2k}\  , \ where\  n\ = 1.2.3\dots\ ,\ k \ \ is \  time \  step
\end{align*}
Recall Eq(8) and $\textbf{u}$ = 
$
\begin{matrix}
\left[
u^{}_{1},\ 
u^{}_{2},\ 
u^{}_{3},\ 
\dots
u^{}_{N}
\right]
\end{matrix}^T             
$
\begin{displaymath}
 \frac{\partial{\textbf{u}}}{\partial{t}}+\frac{\partial{\textbf{u}}}{\partial{x}} = 0 \ ,\ \frac{\partial{\textbf{u}}}{\partial{t}} = - \frac{\partial{\textbf{u}}}{\partial{x}}
\end{displaymath}
Now we have the scheme
\begin{displaymath}
\frac{\textbf{u}^{n+1}_{i}- \textbf{u}^{n-1}_{i}}{2k}= - \frac{3}{h}P^{-1}Q\textbf{u}^{n}
\end{displaymath}

\newpage
\paragraph{Solution 2} : \\

By $\textbf{Solution \ 1}$. we have  the  scheme   
 \begin{displaymath}
\frac{\textbf{u}^{n+1}_{i}- \textbf{u}^{n-1}_{i}}{2k}= - \frac{3}{h}P^{-1}Q\textbf{u}^{n}_{i}
 \end{displaymath}
 \begin{displaymath}
where \  P = 
\left[ 
\begin{matrix}
4 & 1 & ~ & 1\\
1 & \ddots & \ddots & ~\\
~ & \ddots & \ddots & 1\\
1 & ~ & 1 & 4
\end{matrix}
\right] 
,
Q =
\left[
\begin{matrix}
0& 1 & ~ & -1\\
-1 & \ddots & \ddots & ~\\
~ & \ddots & \ddots & 1\\
1 & ~ & -1 & 0
\end{matrix}
\right]
,\textbf{u}^{n}_{i} = \left[ 
\begin{matrix}
u^{n}_{0}\\
u^{n}_{1}\\
u^{n}_{2}\\
\vdots\\
u^{n}_{N}\\
\end{matrix}
\right]
\end{displaymath}
We can translate the shceme  as follows
 \begin{displaymath}
P\textbf{u}^{n+1}_{i} = P\textbf{u}^{n-1}_{i}-  6\lambda Q\textbf{u}^{n}_{i} \ ,\ \lambda =\frac{k}{h}
 \end{displaymath}
Now we focus on $P\textbf{u}^{n+1}_{i}$  
 \begin{displaymath}
P\textbf{u}^{n+1}_{i} = 
\left[
\begin{matrix}
4 & 1 & ~ & 1\\
1 & \ddots & \ddots & ~\\
~ & \ddots & \ddots & 1\\
1 & ~ & 1 & 4
\end{matrix}
\right] \ 
\left[
\begin{matrix}
u^{n+1}_{0}\\
u^{n+1}_{1}\\
u^{n+1}_{2}\\
\vdots\\
u^{n+1}_{N}\\
\end{matrix}
\right]=
\left[
\begin{matrix}
u^{n+1}_{1} + \ 4u^{n+1}_{0}+ \ u^{n+1}_{N}\\
u^{n+1}_{2} + \ 4u^{n+1}_{1}+ \ u^{n+1}_{0}\\
u^{n+1}_{3} + \ 4u^{n+1}_{2}+ \ u^{n+1}_{1}\\
\vdots\\
u^{n+1}_{0}+ \ 4u^{n+1}_{N}+ \ u^{n+1}_{N-1}\\
\end{matrix}
\right]
\end{displaymath}
Similary, we can write $P\textbf{u}^{n-1}_{i}+  6\lambda Q\textbf{u}^{n}_{i}$ as :
 \begin{displaymath}
\left[
\begin{matrix}
u^{n-1}_{1} + \ 4u^{n-1}_{0}+ \ u^{n-1}_{N}\\
u^{n-1}_{2} + \ 4u^{n-1}_{1}+ \ u^{n-1}_{0}\\
u^{n-1}_{3} + \ 4u^{n-1}_{2}+ \ u^{n-1}_{1}\\
\vdots\\
u^{n-1}_{0}+ \ 4u^{n-1}_{N}+ \ u^{n-1}_{N-1}\\
\end{matrix}
\right]
-6\lambda
\left[
\begin{matrix}
u^{n}_{1} - \ u^{n}_{N}\\
u^{n}_{2} - \ u^{n}_{0}\\
u^{n}_{3} - \ u^{n}_{1}\\
\vdots\\
u^{n}_{0} - \ u^{n}_{N-1}\\
\end{matrix}
\right]
\end{displaymath}
Then we have the linear system

\begin{displaymath}
\left[
\begin{matrix}
u^{n+1}_{1} + \ 4u^{n+1}_{0}+ \ u^{n+1}_{N}\\
u^{n+1}_{2} + \ 4u^{n+1}_{1}+ \ u^{n+1}_{0}\\
u^{n+1}_{3} + \ 4u^{n+1}_{2}+ \ u^{n+1}_{1}\\
\vdots\\
u^{n+1}_{0}+ \ 4u^{n+1}_{N}+ \ u^{n+1}_{N-1}\\
\end{matrix}
\right]
=
\left[
\begin{matrix}
u^{n-1}_{1} + \ 4u^{n-1}_{0}+ \ u^{n-1}_{N}\\
u^{n-1}_{2} + \ 4u^{n-1}_{1}+ \ u^{n-1}_{0}\\
u^{n-1}_{3} + \ 4u^{n-1}_{2}+ \ u^{n-1}_{1}\\
\vdots\\
u^{n-1}_{0}+ \ 4u^{n-1}_{N}+ \ u^{n-1}_{N-1}\\
\end{matrix}
\right]
-6\lambda
\left[
\begin{matrix}
u^{n}_{1} - \ u^{n}_{N}\\
u^{n}_{2} - \ u^{n}_{0}\\
u^{n}_{3} - \ u^{n}_{1}\\
\vdots\\
u^{n}_{0} - \ u^{n}_{N-1}\\
\end{matrix}
\right]
\end{displaymath}

Since u(x,t) is 2$\pi$ periodic function , we know 
$
u^{n}_{N} = u^{n}_{-1} , \ 
u^{n}_{0} =u^{n}_{N+1}
$ ,\ we can translate the system as follow :

\begin{displaymath}
\left[
\begin{matrix}
u^{n+1}_{1} + \ 4u^{n+1}_{0}+ \ u^{n+1}_{-1}\\
u^{n+1}_{2} + \ 4u^{n+1}_{1}+ \ u^{n+1}_{0}\\
u^{n+1}_{3} + \ 4u^{n+1}_{2}+ \ u^{n+1}_{1}\\
\vdots\\
u^{n+1}_{N+1}+ \ 4u^{n+1}_{N}+ \ u^{n+1}_{N-1}\\
\end{matrix}
\right]
=
\left[
\begin{matrix}
u^{n-1}_{1} + \ 4u^{n-1}_{0}+ \ u^{n-1}_{-1}\\
u^{n-1}_{2} + \ 4u^{n-1}_{1}+ \ u^{n-1}_{0}\\
u^{n-1}_{3} + \ 4u^{n-1}_{2}+ \ u^{n-1}_{1}\\
\vdots\\
u^{n-1}_{N+1}+ \ 4u^{n-1}_{N}+ \ u^{n-1}_{N-1}\\
\end{matrix}
\right]
-6\lambda
\left[
\begin{matrix}
u^{n}_{1} - \ u^{n}_{-1}\\
u^{n}_{2} - \ u^{n}_{0}\\
u^{n}_{3} - \ u^{n}_{1}\\
\vdots\\
u^{n}_{N+1} - \ u^{n}_{N-1}\\
\end{matrix}
\right]
\end{displaymath}

\begin{displaymath}
\left[
\begin{matrix}
u^{n+1}_{1} \\
u^{n+1}_{2} \\
u^{n+1}_{3} \\
\vdots\\
u^{n+1}_{N+1}\\
\end{matrix}
\right] 
%
+
%
4\left[
\begin{matrix}
u^{n+1}_{0} \\
u^{n+1}_{1} \\
u^{n+1}_{2} \\
\vdots\\
u^{n+1}_{N}\\
\end{matrix}
\right]
%
+
%
\left[
\begin{matrix}
u^{n+1}_{-1} \\
u^{n+1}_{0} \\
u^{n+1}_{1} \\
\vdots\\
u^{n+1}_{N-1}\\
\end{matrix}
\right]
%
=
%
\left[
\begin{matrix}
u^{n-1}_{1} \\
u^{n-1}_{2} \\
u^{n-1}_{3} \\
\vdots\\
u^{n-1}_{N+1}\\
\end{matrix}
\right] 
%
+
%
4\left[
\begin{matrix}
u^{n-1}_{0} \\
u^{n-1}_{1} \\
u^{n-1}_{2} \\
\vdots\\
u^{n-1}_{N}\\
\end{matrix}
\right]
%
+
%
\left[
\begin{matrix}
u^{n-1}_{-1} \\
u^{n-1}_{0} \\
u^{n-1}_{1} \\
\vdots\\
u^{n-1}_{N-1}\\
\end{matrix}
\right]
%
-6\lambda\left(
\left[
\begin{matrix}
u^{n}_{1} \\
u^{n}_{2} \\
u^{n}_{3}\\
\vdots\\
u^{n}_{N+1}\\
\end{matrix}
\right]
%
-
%
\left[
\begin{matrix}
u^{n}_{-1} \\
u^{n}_{0} \\
u^{n}_{1}\\
\vdots\\
u^{n}_{N-1}\\
\end{matrix}
\right]
\right)
\end{displaymath}
\newpage
Let $\textbf{u}^{n}_{i}$
=
$\begin{matrix}
\left[
u^{n}_{0},\ 
u^{n}_{1},\ 
u^{n}_{2},\ 
\dots
u^{n}_{N}
\right]
\end{matrix}^T$ , \ we have
\begin{equation}
\textbf{u}^{n+1}_{i+1}+ 4\textbf{u}^{n+1}_{i}+\textbf{u}^{n+1}_{i-1}
=
\textbf{u}^{n-1}_{i+1}+ 4\textbf{u}^{n-1}_{i}+\textbf{u}^{n-1}_{i-1}
-6\lambda
(\textbf{u}^{n}_{i+1}-\textbf{u}^{n}_{i-1})
, \ i =0,\ 1,\ 2,\ ,3\ \dots, \ N
\end{equation}    
Since u(x,t) is 2$\pi$ periodic, we assume the solution of $u(x^{}_{i},t^{}_{n})$ is 
\begin{align*}
u(x^{}_{i},t^{}_{n}) =u^{n}_{i} =\ \hat{u}^n\ \cdot \ e^{i \omega x^{}_{i}},\ i = \sqrt{-1},\ \omega \in \textbf{Z} 
\end{align*}

Then we can rewrite Eq(12) as

\begin{displaymath}
\hat{u}^{n+1}e^{i \omega x^{}_{i}}(e^{i \omega h}+4+e^{-i \omega h}) =
\hat{u}^{n-1}e^{i \omega x^{}_{i}}(e^{i \omega h}+4+e^{-i \omega h})-\ 6\lambda\hat{u}^{n}e^{i \omega x^{}_{i}}(e^{i \omega h}-  e^{-i \omega h}) \\
\end{displaymath}
Divided by $e^{i \omega x^{}_{i}}$ 
\begin{displaymath}
\hat{u}^{n+1} (e^{i \omega h}+4+e^{-i \omega h})=  \hat{u}^{n-1}(e^{i \omega h}+4+e^{-i \omega h}) - 6\lambda\hat{u}^{n}(e^{i \omega h}-  e^{-i \omega h})
\end{displaymath}












We know $e^{i \omega h} = \cos(\omega h) + i \sin(\omega h)$ so, 
\begin{equation}
\hat{u}^{n+1} (\cos(\omega h) + 2)=  \hat{u}^{n-1}(\cos(\omega h) + 2) - 6\lambda\hat{u}^{n}(i \sin(\omega h))
\end{equation}
Since  $-1 \le \cos(\omega h) \le 1$ , $\cos(\omega h) + 2 \neq 0$ ,we can write Eq(13) as
\begin{equation}
\hat{u}^{n+1} =  \hat{u}^{n-1} - \hat{u}^{n}
\frac{6\lambda i \sin (\omega h)}{\cos(\omega h) + 2}
\end{equation}
Rewrite Eq(14) as a linear system :
\begin{equation}
\left[
\begin{matrix}
\hat{u}^{n+1} 
\\
\hat{u}^{n} 
\end{matrix}
\right]
=
\left[
\begin{matrix}
 -\frac{6\lambda i \sin (\omega h)}{\cos(\omega h) + 2}& 1
\\
1  & 0
\end{matrix}
\right]
%
\left[
\begin{matrix}
\hat{u}^{n} 
\\
\hat{u}^{n-1} 
\end{matrix}
\right]
\end{equation}

Let 
\begin{displaymath}
A =
\left[
\begin{matrix}
 -\frac{6\lambda i \sin (\omega h)}{\cos(\omega h) + 2}& 1
\\
1  & 0
\end{matrix}
\right]
\end{displaymath}
Then we can write Eq(15) as
\begin{equation}
\left[
\begin{matrix}
\hat{u}^{n+1} 
\\
\hat{u}^{n} 
\end{matrix}
\right]
=
A
%
\left[
\begin{matrix}
\hat{u}^{n} 
\\
\hat{u}^{n-1} 
\end{matrix}
\right]
\end{equation}
Assume 
\begin{displaymath}
\hat{V}^n =
\left[
\begin{matrix}
\hat{u}^{n+1} 
\\
\hat{u}^{n} 
\end{matrix}
\right]
, n =0,\  1,\ 2,\ 3,\ \dots
\end{displaymath}
Now we have
\begin{align*}
\hat{V}^n & =A\hat{V}^{n-1}\\
&= A(A\hat{V}^{n-2})\\
& = AA(A\hat{V}^{n-3})\\
& = \ \ \ \ \vdots\\
& = \ \ \ \ \vdots\\
& = A^n\hat{V}^{0}
\end{align*}
\newpage
We recall the hypothetical solutoion of $u(x^{}_{i},t^{}_{n})$
\begin{displaymath}
u(x^{}_{i},t^{}_{n}) = u^{n}_{i} =\ \hat{u}^n\ \cdot \ e^{i \omega x^{}_{i}} ,\  \omega \in \textbf{Z} 
\end{displaymath}

where  $t^{}_{n}$ = n $\cdot$ k, k is time step\\
\\
When $t^{}_{n} $is fixed , we let $k \rightarrow 0$,then$\ n \rightarrow \infty$ .

\begin{displaymath}
\hat{V}^n =
\left[
\begin{matrix}
\hat{u}^{n+1} 
\\
\hat{u}^{n} 
\end{matrix}
\right]
= A^n\hat{V}^{0}
\end{displaymath}

We want $\hat{u}^{n}$ exits, namely , $\hat{V}^n$ is converge ,
$A^{n}$ must have upper bound
\\
we let b = $\frac{\sin (\omega h)}{\cos(\omega h) + 2}$ ,\ and $\vert b\vert <\ \frac{1}{3} $
\\

\begin{displaymath}
A =
\left[
\begin{matrix}
 -6\lambda i b& 1
\\
1  & 0
\end{matrix}
\right]
\end{displaymath}











\end{document}